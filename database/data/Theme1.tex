\textbf{Téma 1: Úvod do Matematickej štatistiky}

Matematická štatistika je odbor matematiky zaoberajúci sa zberom, analýzou a interpretáciou dát. Jej hlavným cieľom je odhadovať vlastnosti populácie na základe informácií obsiahnutých vo vzorke. Popisná štatistika sa zameriava na zhrnutie dát pomocou mier centrálnej tendencie a variability. Teória pravdepodobnosti je základom matematickej štatistiky a umožňuje kvantifikovať neistotu v rôznych javoch. Inferenčná štatistika poskytuje metódy na odvodzovanie záverov o celej populácii na základe dát získaných z jej vzorky. Regresná analýza skúma vzťahy medzi premennými a umožňuje predikciu hodnôt na základe znalostí iných premenných. Bayesovská štatistika využíva Bayesovu vetu na aktualizáciu pravdepodobností na základe nových dôkazov. Matematická štatistika má široké uplatnenie v rôznych oblastiach, vrátane vedeckého výskumu, ekonómie a zdravotníctva. Je mocným nástrojom pre interpretáciu dát a rozhodovacie procesy. Pomáha vytvárať informované rozhodnutia a porozumieť javom vo svete okolo nás.

\textbf{Príklad: Hádzanie mincí}

Predstavme si, že chceme zistiť pravdepodobnosť toho, že pri jednom hode mincí padne hlava. Etapy štatistickej práce:

\begin{enumerate}
    \item \textbf{Zber dát}: Vykonáme sériu hodov mincí. Napríklad 100 hodu mincí.
    \item \textbf{Spracovanie dát}: Po vykonaní 100 hodov mincí zistíme, že v 60 prípadoch padla hlava.
    \item \textbf{Analýza dát}: Teraz môžeme vypočítať pravdepodobnosť toho, že padne hlava. Stačí vydeliť počet prípadov, keď padla hlava (60), celkovým počtom hodov (100). Teda pravdepodobnosť hlavy je:
    \[
        P(\text{hlava}) = \frac{60}{100} = 0.6 \quad \text{(60\%)}
    \]
    \item \textbf{Záver}: Na základe nášho experimentu sme zistili, že pravdepodobnosť, že pri hode mincí padne hlava, je približne 60\%. Toto je odhad pravdepodobnosti získaný z našej vzorky.
\end{enumerate}

\textbf{Príklad: Oslava a pravdepodobnosť mien}

Na oslave je 100 hostí, z ktorých 10 má meno Martin a 15 z nich majú meno, ktoré začína na písmeno F. Každý z nich má práve jedno meno.

\begin{itemize}
    \item a1) Aká je pravdepodobnosť, že sa bude volať Martin?
    \item a2) Aká je pravdepodobnosť, že meno hosťa bude Martin, alebo sa bude začínať na písmeno F?
    \item a3) Aká je pravdepodobnosť, že meno hosťa nebude Martin, ale bude začínať na písmeno F?
    \item a4) Aká je pravdepodobnosť, že meno hosťa nebude Martin a nebude sa začínať na písmeno F?
    \[
        1 - 0.25 = 0.75
    \]
\end{itemize}

\textbf{Hádzanie kockou}

Strany kocky sú označené 1 až 6.

\begin{itemize}
    \item Aká je pravdepodobnosť, že spadne číslo 3? \[
                                                         \frac{1}{6}
    \]
    \item Aká je pravdepodobnosť, že spadne číslo väčšie ako 4? \[
                                                                    \frac{2}{6} = \frac{1}{3}
    \]
    \item Aká je pravdepodobnosť, že spadne číslo menšie ako 5? \[
                                                                    \frac{4}{6} = \frac{2}{3}
    \]
\end{itemize}

\textbf{Medián}

Medián označujeme \textit{med(x)}. Je to prostredný člen spomedzi hodnôt \( x_i \), ak sú usporiadané podľa veľkosti. Ak je rozsahom súboru párne číslo \( n \), medián určíme nasledovne:
\[
    \text{med}(x) = \frac{x_i + x_{i+1}}{2}, \quad i = \frac{n}{2}
\]
Tento výpočet predstavuje aritmetický priemer „prostredných“ dvoch členov.

\textbf{Modus}

Modus označujeme \textit{mod(x)}. Je to najčastejšie sa vyskytujúca hodnota v súbore.

\textbf{Príklad: Známky žiakov 6. A}

Žiaci 6. A dostali z písomnej práce z matematiky nasledovné známky:
\[
    3, 2, 1, 1, 2, 2, 3, 4, 4, 1, 2, 3, 3, 3, 5
\]

\textbf{Medián}:
\[
    \text{med} = \frac{(3 + 2 + 1 + 1 + 2 + 2 + 3 + 4 + 4 + 1 + 2 + 3 + 3 + 3 + 5)}{2} = 2.6
\]

\textbf{Modus}: 3 (najčastejšie vyskytujúca sa hodnota)
